\documentclass[paper=a4, fontsize=11pt]{scrartcl}
\usepackage{fourier}
\usepackage[table,xcdraw]{xcolor}
\usepackage[protrusion=true,expansion=true]{microtype}	
\usepackage{amsmath,amsfonts,amsthm} % Math packages
\usepackage{graphicx}	
\usepackage{url}
\usepackage{bookmark}

\usepackage{polyglossia}
\setdefaultlanguage{greek}
\setotherlanguages{english}
%%% Bibliography 
\usepackage{biblatex}
\addbibresource{ref.bib}


\usepackage{float}
\restylefloat{table}
%%%% algorithms
\usepackage{algorithm} 
\usepackage{algpseudocode} 


\graphicspath{ {./images/} }


%%%fonts
\usepackage{fontspec}
\setmainfont{FreeSerif}
\setsansfont{FreeSans}
\setmonofont{FreeMono}


\usepackage{sectsty}
\allsectionsfont{\centering \normalfont\scshape}


%%% Custom headers/footers (fancyhdr package)
\usepackage{fancyhdr}
\pagestyle{fancyplain}
\fancyhead{}											% No page header
\fancyfoot[L]{}											% Empty 
\fancyfoot[C]{}											% Empty
\fancyfoot[R]{\thepage}									% Pagenumbering
\renewcommand{\headrulewidth}{0pt}			% Remove header underlines
\renewcommand{\footrulewidth}{0pt}				% Remove footer underlines
\setlength{\headheight}{13.6pt}


\numberwithin{equation}{section}		% Equationnumbering: section.eq#
\numberwithin{figure}{section}			% Figurenumbering: section.fig#
\numberwithin{table}{section}				% Tablenumbering: section.tab#

%%%paragraph 
\setlength{\parindent}{1em}
\setlength{\parskip}{0.5em}

\usepackage{titlesec, lipsum}
\titleformat{\paragraph}[block]{\filcenter}{}{0pt}{}

%%% Maketitle metadata
\newcommand{\horrule}[1]{\rule{\linewidth}{#1}} 	% Horizontal rule




\title{
     \usefont{OT1}{bch}{b}{n}
     \normalfont \normalsize \textsc{Πανεπιστήμιο Ιωαννίνων Τμήμα Πληροφορικής και Τηλεπικοινωνίων} \\ [1em]
     \normalfont \normalsize \textsc{Αλγόριθμοι και πολυπλοκότητα} \\ [1em]
     \horrule{0.5pt} \\[0.4cm]
     \huge Δημιουργία λαβυρίνθου, σχεδίαση διαδρομής από την είσοδο στην έξοδο. \\
     \horrule{2pt} \\[0.5cm]
}

\author{
    \normalfont
         Παναγιώτα Βίννη  Α.Μ. 1873\\[-1pt]
         \today
}

\date{}

\begin{document}

\maketitle

\section*{Περίληψη}
   Σε αυτήν την τεχνική αναφορά εξετάζεται η δημιουργία και η σχεδίαση ενός λαβύρινθου,ενώ χρησιμοποιείται και ο αλγόριθμος BFS (Breadth-first search) για την συντομότερη επίλυση του.
   
\section{Εισαγωγή}
    Το πρόβλημα του λαβύρινθου είναι ένα πρόβλημα στο
    οποίο δίνεται μήκος, ύψος, είσοδο και έξοδος (θετικές ακέραιες τιμές) και ζητείται να βρεθεί μια σειρά από ακμές που θα οδηγει από την είσοδο στην έξοδο με τον πιο συντομότερο τρόπο.
    
    Για τη σχεδίαση του χρησιμοποιήθηκε δομή Union - Find και τα disjoint sets όπου στις ήδη αφαιρεμένες κορυφές βρίσκει τις κορυφές εκείνες που δεν έχουν επισκεφτεί και τις προσθέτει στις removed edges, ενώνοντας τες όλες μάζι για να δημιουργηθεί ένα μονοπάτι. Το μόνο απαραίτητο όμως είναι να ανήκει η είσοδος και η έξοδος στο ίδιο σύνολο.
    
    

    
\section{Αποτελέσματα}
    Για την συγγραφή της εργασίας χρησιμοποιήθηκε η Python 3.9.2 και το Visual Studio Code. Τα χαρακτηριστικά του υπολογιστή είναι Intel Core i7-1065G7 (1.30GHz - 1.50 GHz), 8 GB DDR3 (2667 MHz), λειτουργικό σύστημα Windows 10 Home x64. Για την οπτικοποίηση των αποτελεσμάτων χρησιμοποιήθηκε η βιβλιοθήκη matplotlib. Στον πίνακα ~\ref{tab:results} φαίνονται τα αποτελέσματα του κώδικα για τον λαβύρινθο 3x3 και στην εικόνα ~\ref{fig:maze} φαίνεται η δημιουργία του λαβύρινθου 3x3 με την εκτέλεση του κώδικα. Τέλος, οδηγίες για την εκτέλση του κώδικα: python maze.py


 \centering
\begin{table}[htbp]
  \caption{Πίνακας αποτελεσμτάτων}
    \begin{tabular}{ll} \\
    \textbf{Length} & 3 \\
    \textbf{Height} & 3 \\
    \textbf{Entrance} & UP,1 \\
    \textbf{Exit} & RIGHT,9 \\
    \textbf{Adjacency list} & {(5, 8), (1, 4), (2, 3), (4, 5), (8, 9), (5, 6), (2, 5), (4, 7)} \\
    \textbf{Adjacency map} & {5: {8, 2, 4, 6}, 
                              8: {9, 5}, 
                              1: {4}, 
                              4: {1, 5, 7}, 
                              2: {3, 5}, 
                              3: {2}, 
                              9: {8}, 
                              6: {5}, 
                              7: {4}} \\
    \textbf{Path with BFS} & [1, 4, 5, 8, 9] \\
    \end{tabular}%
  \label{tab:results}%
\end{table}%


\begin{figure}[h]
    \caption{Λαβύρινθος 3x3.}
    \centering
    \includegraphics[width=10cm, height=7cm]{maze.png}
    \label{fig:maze}
\end{figure}  


    
\section{Συμπεράσματα}
      Η σχεδίαση λαβυρίνθων είναι ένα ενδιαφέρον πρόβλημα που υλοποιείται κάνοντας χρήση της δομής ξένων συνόλων (Disjoint Sets), ενώ για την εύρεση του συντομότερου μονοπατιού από την είσοδο μέχρι την έξοδο έγινε χρήση του αλγορίθμου BFS (Breadth-First Search).
    
    

\end{document}
